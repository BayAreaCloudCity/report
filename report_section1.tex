\chapter{Business Analysis}

The IoT market in the U.S. is expected to increase from approximately \$118.2 billion in 2023 to about \$553.9 billion in 2030 at a compound annual growth rate (CAGR) of 24.7\%. This forecast underscores the rapid expansion of the IoT market \citet{fortune2020us}. As the IoT market evolves, IoT networks facilitate digital twins, which simulate real-world systems and processes in virtual environments. This advancement enhances the monitoring and forecasting of physical operations across multiple sectors. However, creating digital twins requires a profound understanding of complex data engineering, software engineering, and cloud infrastructure. Furthermore, according to \citet{trotino2021building}, data scientists currently spend about 80\% of their time building data pipelines for data analysis, leaving only 20\% for analysis and other substantive tasks. 

Our project proposes a novel methodology for developing a digital twin of full-stack applications using a low-code framework, which does not require specialized knowledge from users such as data scientists. 
Specifically, we developed a real-time traffic forecasting and analysis model for a designated segment of the I-880 freeway. We integrated data from speed sensors, weather conditions, and traffic operation events within a cloud environment to train a predictive model. This model was then utilized to forecast traffic flow 10 minutes into the future based on real-time data, with the predictions displayed on an intuitive dashboard. Moreover, this project explores the potential of low-code approaches by reverse-engineering the developed models and incorporating the ensuing reference architecture and domain-specific language. This process aims to streamline the development of digital twin applications, making them accessible even to users with limited data engineering and software engineering expertise.

\newpage

\section{Traffic Analysis}
\subsubsection{Traffic Congestion Factors}
The Cambridge \citet{systematics2005traffic} report outlines current conditions, trends, and countermeasures for traffic congestion in the United States, focusing on traffic flow-affecting factors. It details seven significant causes, including physical bottlenecks, traffic accidents, construction zones, weather, traffic control devices, special events, and normal traffic volume fluctuations. These factors have similarities and differences with other machine learning-based traffic flow forecasting research. Appropriately considering these key factors affecting traffic flow when forecasting traffic flow using machine learning can provide valuable insights to achieve more practical forecasts grounded in reality.

\subsubsection{Traffic Flow Prediction Models}
\citet{garrett2020integrated} developed a machine learning-based traffic forecasting package that considers weather, construction sites, accidents, and special events. They integrate actual traffic data with external factors and apply decision trees and Markov models to offer a practical tool for predicting traffic congestion. Meanwhile, \citet{zafar2020traffic} discusses the accuracy of machine learning methods such as Random Forest, KNN, XGBoost, and GradientBoost, with the highest accuracy reaching 92\%, indicating the potential for improvement. Additionally, \citet{park2011real} presents a neural network-based speed prediction algorithm that uses current traffic information. These studies demonstrate the effectiveness of combining real-time data with external factors and applying machine learning and deep learning techniques to traffic flow forecasting, providing essential insights for building more realistic traffic management models.

\subsubsection{Road Segmentation}
In forecasting traffic flows, it is not realistic to uniformly forecast the entire roadway since, in reality, only a portion of the roadway is often very congested. Therefore, the guidelines in the Highway Capacity Manual by \citet{bob2023freeway} provide a method for forecasting traffic flow that does not uniformly forecast the entire roadway but analyzes specific segments separately, such as merging points, turnouts, lane numbers, and sections with different speed limits. This approach makes our project more realistic and detailed traffic flow forecasts.

\section{Cloud-Based Technologies}
\subsubsection{Streaming Data}
Creating a function to periodically collect data and pass it as streaming data to our pipeline is straightforward. However, \citet{damji2020learning} highlight the challenge of streaming data and proposes a solution. The issue is that data or messages might not be delivered to our system on time due to network connectivity or the iterative characteristics of data collection. He suggests defining a time window to discard any out-of-order data delayed beyond this window. This consideration is crucial when replaying data for simulation, addressing delay messages, and managing these situations in our online streaming prediction.

\subsubsection{Event-Driven Applications}
For smart city applications, updates often come in the form of events from diverse sources. Hence, event-driven development is a popular choice for these applications. \citet{hinze2009event} suggest that traffic monitoring, which involves gathering information from numerous sensors, could utilize this method. Although no realized applications were provided, a publisher/subscriber system for event communication and data mining for traffic pattern insights was recommended.

\subsubsection{Digital Twins for Smart Cities}
Digital Twins replicate physical world signals or data in the digital realm. For transportation, analyzing speed data from freeway sensors allows transportation departments to make more informed and timely decisions \citet{hu2021digital}. Thus, the digital twins concept involves collecting ample real-world data, extracting useful information, and making data-driven decisions across various fields.

\section{Low-Code Solutions}
\subsubsection{Low-Code Platforms}
Low-code platforms have gained popularity, enabling individuals with minimal coding expertise to develop applications swiftly. \citet{sahay2020supporting} provide a comprehensive comparison of existing low-code platforms, such as Google App Maker and Salesforce. They observed that these platforms share similar design models but lack built-in AI support, advanced business insights reporting, and support for event-driven applications.

\subsubsection{Anaximander Framework}
The Anaximander framework is a Python library that combines object-oriented programming with data science tools. The framework is designed to provide concise declarations of data, metadata, and transformation pipelines. Anaximander also automates the setup, configuration, and management of infrastructure, software, systems, and the resources and services required for data access in a cloud environment. This allows developers to focus on business logic and data science-related issues. In addition, the Anaximander framework supports event-driven applications through the use of low-code technologies \citet{jd2021anaximander}. It models data sources with Python code, enabling developers to manage them as traditional Python objects through its object-oriented design. This facilitates expressive programming and allows for quick testing of new ideas and accelerated innovation, although it has not yet been applied in real-world applications.

\section{Decision Support Systems}
\subsubsection{Dashboard for Data-driven Decision Making}
Smart city applications assist users in making data-driven decisions. Therefore, a dashboard that visualizes and summarizes data can be beneficial. \citet{tempelmeier2020ta} propose the Traffic Analytics Dashboard (TA-Dash), an interactive dashboard for visualizing urban traffic patterns over time and space. The usefulness of TA-Dash is demonstrated through showcased by illustrating how it can analyze, predict, and visually represent the effects of special events on traffic. This insight underscores the need for a dashboard to visualize model predictions on geometric maps and line plots, enabling even non-expert users to understand the results.
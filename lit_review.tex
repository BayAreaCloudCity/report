\documentclass[12pt]{article}

\usepackage{setspace}
\usepackage{hyperref}
\title{ENGIN 295 Literature Review Annotated Bibliography}
\author{Team 24 Cloud City (Qingyang Hu, Jeff Liu, Nozomu Kitamura)}
\date{}
\begin{document}
\setlength{\parindent}{-40pt}

\maketitle
JD Margulici. 2022. ``Anaximander: The Rapid Application Development Framework for Data-intensive Python.'' {\it Novavia Solutions}. 
This document describes the development background and scope of the Anaximander framework. Anaximander is a Python library that combines object-oriented programming and data science tools to enable concise declarations of data, metadata, and transformation pipelines. In addition, the Anaximander framework automatically automates setting up, configuring, and managing the resources and services required by infrastructure software and systems within a cloud environment and data access, allowing developers to focus on business logic and data science problems. \\

Cambridge Systematics. 2005. ``Traffic Congestion and Reliability: Trends and Advanced Strategies for Congestion Mitigation.'' {\it US Department of Transportation Federal Highway Administration}. \url{https://rosap.ntl.bts.gov/view/dot/20656}.
This report outlines the current state of traffic congestion in the United States, recent trends, and efforts to curb the increase in congestion. Focusing on the variables that affect traffic flow, the report details seven significant causes of traffic congestion, including physical bottlenecks, traffic accidents, construction sites, weather, traffic control devices, special events, and routine traffic volume changes. \\

J. Kyle Garrett, Jiaqi Ma, Hani Mahmassani, Michelle Neuner, and Bob Sanchez. 2020. ``Integrated Modeling for Road Condition Prediction Phase 3 Project Report.'' {\it US Department of Transportation Federal Highway Administration}. \url{https://ops.fhwa.dot.gov/publications/fhwahop20061/fhwahop20061.pdf}.
In this study, a system configuration and operational framework were developed to forecast traffic and road weather conditions using the Integrated Modeling System (IMRCP). The Machine Learning Based Traffic Prediction (MLP) package within it predicts traffic conditions, considering weather, construction zones, accidents, and special events. This research provides valuable insight into functional implementation as a practical tool for accurately capturing and predicting traffic congestion by integrating real-world traffic data with external environmental factors. This resource is similar to Noureen Zafar’s paper in that it utilizes various data, including weather data, to create machine-learning models. \\

Noureen Zafar and Irfan U. Haq. 2020. ``Traffic Congestion Prediction Based on Estimated Time of Arrival.'' {\it PLOS ONE} 15, no 12. \url{https://doi.org/10.1371/journal.pone.0238200}.
This paper discusses a traffic congestion forecasting methodology using data collected from Google Maps. The study uses Estimated Time of Arrival (ETA) as its core and combines various factors such as weather, time of day, special events, and holidays to predict traffic conditions. The study provides an approach to traffic forecasting that considers multiple factors and provides valuable insights for traffic management in urban areas such as this project. This is similar to J. Kyle Garrett’s paper because the model is created using various data, including weather data, to create a model to predict traffic conditions. \\

Jungme Park, Dai Li, Yi L. Murpheym, Johannes Kristinsson, Ryan McGee, Ming Kuang, and Tony Phillips. 2011. ``Real time vehicle speed prediction using a Neural Network Traffic Model.'' {\it The 2011 International Joint Conference on Neural Networks}, San Jose, CA, USA. \url{https://ieeexplore.ieee.org/abstract/document/6033614}.
This paper examines various methods of traffic information prediction based on traffic flow, density, and speed. A speed prediction algorithm, NNTM-SP (Neural Network Traffic Modeling-Speed Prediction), which utilizes neural networks trained with historical traffic data and can predict vehicle speed profiles from current traffic information, is presented. This paper offers new insights into using neural networks compared to Noureen Zafar’s and J. Kyle Garrett’s paper. \\

McTrans Center. 2023. ``Freeway Facilities – HCM Segmentation Process.'' {\it Youtube}. \url{https://www.youtube.com/watch?v=3A9SPRCnUHs}.
To achieve efficient traffic flow analysis, this video presents a method of dividing the roadway into multiple segments, following the guidelines of the Highway Capacity Manual (HCM). This approach allows specific segments, such as merges, junctions, changes in the number of lanes, and sections with different speed limits, to be analyzed individually rather than forecasting the entire roadway uniformly. Applying this approach to a project allows for more detailed and precise traffic flow forecasting. \\

A. Sahay, A. Indamutsa, D. Di Ruscio and A. Pierantonio. 2020. ``Supporting the understanding and comparison of low-code development platforms.'' {\it 46th Euromicro Conference on Software Engineering and Advanced Applications (SEAA)}, Portoroz, Slovenia. \url{https://ieeexplore.ieee.org/abstract/document/9226356/}.
This paper provides a technical survey of low-code development platforms that enable software development for users without various programming experiences. The paper also selects eight representative LCDPs and analyzes each based on a conceptual comparative framework to facilitate understanding and comparing each low-code platform. \\


Heitor Murilo Gomes, Jesse Read, Albert Bifet, Jean Paul Barddal, and João Gama. 2019. ``Machine learning for streaming data: state of the art, challenges, and opportunities.'' {\it SIGKDD Explor. Newsl.} 21, 2.
\url{https://dl.acm.org/doi/pdf/10.1145/3373464.3373470}.
The paper focuses on learning to update models from a continuous influx of data without multiple passes through the data, i.e., algorithms such as incremental learning, online learning, and data stream learning. The focus is on identifying the current state of the art in related areas and identifying outstanding issues regarding the speed and volume that characterize big data processing, as many challenges need to be resolved before existing methods can be efficiently applied to real-world problems. \\

Annika Hinze, Kai Sachs, and Alejandro Buchmann. 2009. ``Event-based applications and enabling technologies.'' In {\it Proceedings of the Third ACM International Conference on Distributed Event-Based Systems} (DEBS '09). Association for Computing Machinery, New York, NY, USA, Article 1, 1–15. \url{https://dl.acm.org/doi/pdf/10.1145/1619258.1619260}.
This dissertation focuses on event processing techniques. Since there are many different approaches to event processing, the authors introduce the basic concepts of event processing and survey the techniques used to build event-based systems and the applications of event processing in various applications. They also discuss unsolved problems in the field and areas needing further research. \\



Nicolas Tempelmeier, Anzumana Sander, Udo Feuerhake, Martin Löhdefink, and Elena Demidova. 2020. ``TA-Dash: An Interactive Dashboard for Spatial-Temporal Traffic Analytics.''
In {\it Proceedings of the 28th International Conference on Advances in Geographic Information Systems} (SIGSPATIAL '20). Association for Computing Machinery, New York, NY, USA.
\url{https://dl.acm.org/doi/abs/10.1145/3397536.3422344}.
To increase the effectiveness of complex machine learning models for predicting urban traffic and its impact on road infrastructure, this thesis presents the Traffic Analytics Dashboard (TA-Dash), an interactive dashboard for visualizing urban traffic patterns over space and time. The usefulness of TA-Dash is demonstrated through examples of analyzing, predicting, and visualizing the impact of special events on urban road traffic and analyzing and visualizing structural dependencies within urban road networks.





\end{document}
\chapter{Executive Summary}
\integrity{NK}{QH} This project introduces a low-code framework that enables the development of digital twins without deep expertise in data engineering, software engineering, and cloud infrastructure, leveraging advancements in IoT technology. Specifically, we developed a traffic forecasting model for a specific segment of the I-880 freeway, integrating diverse real-time data sources such as speed sensors, weather conditions, and traffic event information. This model can predict traffic flow 10 minutes into the future based on real-time data, with the predictions visualized on an intuitive dashboard. Furthermore, the project explored the potential of the low-code framework by reverse-engineering the developed application into a reference architecture and domain-specific language (DSL). This low-code framework simplifies the digital twin development process by automatically generating application code from the DSL, utilizing the reference architecture we explored. Future work will focus on implementing the low-code framework's functionality, including converting DSL into executable code and creating user-friendly management tools. In addition, efforts will be made to enhance the accuracy of our predictive models through more detailed redefinition of event severity, improved feature engineering and selection processes, and the exploration of suitable deep-learning architectures for time series forecasting. The reference architecture will also need updates to incorporate components supporting continuous improvement, locally or in the cloud. These efforts aim to make digital twin development more accessible to a broader audience and enable the creation of more accurate and practical predictive models.

\wordcount{report_executive_summary}
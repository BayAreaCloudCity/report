\documentclass[12pt]{article}

\usepackage{setspace}
\usepackage[margin=1in]{geometry}

\doublespacing

\title{ENGIN 295 Literature Review Planning Questions}
\author{Team 24 Cloud City (Qingyang Hu, Jeff Liu, Nozomu Kitamura)}
\date{}
\begin{document}

\maketitle
\begin{enumerate}
    \item \textbf{What would you say your field is? If you were to imagine presenting your project as part of a themed poster session at a field-focused conference, what other projects could you imagine appearing in the room with yours?  How would yours be different?}

\begin{itemize}
    \item Fields: Cloud Computing, Machine Learning, Transportation, Declarative Code Generation.
    \item Other projects: Traffic Optimization using Machine Learning and Data Analysis.
    \item Project Differentiator: Our project features a unique low-code approach, utilizing the Anaximander framework and deployed on a cloud platform. This approach sets it apart from others, as it efficiently integrates multiple data sources, thereby supporting users in developing large-scale applications quickly.
\end{itemize}


    \item \textbf{How would you describe the use case for your project? If the project is not strictly academic, would it make sense for your lit review to discuss business elements like growth forecasts or other industry trends?  What other comparisons (business or otherwise) might you offer between your project and others?}

    \begin{itemize}
        \item Use Case: The primary goal is to solve the specific challenge of optimizing traffic flow in smart cities. This involves integrating various data sources to analyze and predict traffic conditions in real-time, thereby helping to optimize traffic management.
        \item Business Element: Nowadays, the proliferation of sensors across numerous settings is remarkable, and the field of leveraging data for operational optimization is growing at an impressive rate.
        \item Comparison: A key differentiator is the innovation and efficiency in the ability to automatically integrate data using a low-code approach and display it on a dashboard.
    \end{itemize}






 
    \item \textbf{What counts as “literature” for your literature review? Likewise, what kind of person or entity counts as an authority or expert in your field?}

    \begin{itemize}
        \item Literature: This includes academic papers, case studies, and industry reports on topics such as transportation engineering and machine learning.
        \item Experts: Authorities in this field include university professors and industry experts specializing in transportation, smart city technology, and data science.
    \end{itemize}


 
    \item \textbf{Where do your project’s core ideas and/or your methodology come from? Apart from your adviser, who does this kind of work?  What are major names in the field?  If possible, cite a few titles of key sources here.}

\begin{itemize}
    \item Cloud Computing and Template-driven Code Generation: A user-declared and automated method for collecting, processing, and analyzing large data sets was introduced by the mentor. This is essential for understanding how to integrate complex and varied urban data to build predictive models.
    \item Building Smart Cities by Optimizing Traffic Flows: After discussions with the mentor and the entire team, we decided to pursue an approach focused on building smart cities by predicting and optimizing urban traffic flows.
\end{itemize}



    \item \textbf{What work makes you think that the method you’re planning to use to answer your driving question is the best way to do so (i. e., whose work are you drawing on)? If you’re following your adviser’s directions, what makes your adviser think this is the right methodology?  If your project largely combines or reapplies existing techniques in new ways, whose work would suggest that such a combination or application might succeed?}

\begin{itemize}
    \item Prior Research on Traffic Forecasting: References should be made to prior research on existing traffic flow forecasting models. This includes machine learning regression models, deep learning-based approaches, etc.
    \item Data Integration Techniques: The project will draw from studies using the existing Anaximander framework and other applications. Particular attention will be paid to methods for integrating heterogeneous data and extracting meaningful insights.
\end{itemize}





\end{enumerate}



\end{document}